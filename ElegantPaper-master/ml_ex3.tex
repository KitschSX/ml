	%!TEX program = xelatex
% 完整编译: xelatex -> bibtex -> xelatex -> xelatex
\documentclass[lang=cn,10pt,a4paper,cite=authoryear]{elegantpaper}

\title{统计学习实验三:Logistics Regression}
\author{王嗣萱 2018110601014}
\date{}


% 本文档命令
\usepackage{array}
\newcommand{\ccr}[1]{\makecell{{\color{#1}\rule{1cm}{1cm}}}}

\begin{document}

\maketitle

\section{实验原理}

\subsection{Logistic 分布}
Logistic 分布是一种连续型的概率分布,其分布函数和密度函数分别为:

$F(x) = P(X \leq x)=\frac{1}{1+e^{-(x-\mu)/\gamma}} $

$ f(x) = F^{'}(X \leq x)=\frac{e^{-(x-\mu)/\gamma}}{\gamma(1+e^{-(x-\mu)/\gamma})^{2}}$

其中, $\mu$ 表示位置参数, $\gamma$ 为形状参数。我们可以看下其图像特征:

\begin{figure}[htbp]
	\centering
	\includegraphics[width=0.3\textwidth]{lo_1.png}
	\includegraphics[width=0.3\textwidth]{lo_2.png}
	\caption{Logistic 分布}
\end{figure}

Logistic 分布是由其位置和尺度参数定义的连续分布。Logistic 分布的形状与正态分布的形状相似,但是 Logistic 分布的尾部更长,所以我们可以使用 Logistic 分布来建模比正态分布具有更长尾部和更高波峰的数据分布。在深度学习中常用到的 Sigmoid 函数就是 Logistic 的分布函数在  $\mu=0, \gamma=1$ 的特殊形式。

\subsection{Logistic 回归}
本次实验中考虑'0','1'的二分类问题,即一个伯努利分布

$P(y=1|x;\theta)=h_{\theta}(x) $

$P(y=0|x;\theta)=1-h_{\theta}(x)$

可以将其合并为一个表达式:

$P(y|x;\theta)=(h_{\theta}(x))^{y}(1-h_{\theta}(x))^{1-y}$

logistic regression的目标函数是根据最大似然思想求得的。似然函数为:

$L(\theta)=\prod_{i=1}^{n}(h_{\theta}(x^{i}))^{y^{i}}(1-h_{\theta}(x^{i}))^{1-y^{i}}$

对$L(\theta)$求对数可以得到:

$l(\theta)=-logL(\theta)=-\sum_{i=1}^{n}[{y^{i}}log(h_{\theta}(x^{i}))+(1-y^{i})log(1-h_{\theta}(x^{i}))]$

使用$J(\theta)=\frac{1}{m}l(\theta)$作为logistic regression的损失函数



\subsection{}

\section{Python代码实现}

\section{结果及图形展示}

\section{总结体会}

\end{document}
